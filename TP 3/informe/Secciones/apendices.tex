\subsection{Apéndice A : Algoritmos}

\subsubsection{Algoritmo para eliminar Frames Intermedios} \label{eliminarframes}

\lstset{language=C++, breaklines=true, basicstyle=\footnotesize}
\begin{lstlisting}[frame=single]
void eliminarFrames (Parametros &params){

	assert(params.framesIntermedios < (params.frames -2));

	params.output << (params.frames/(params.framesIntermedios+1)) + ( (params.frames % (params.framesIntermedios+1) != 0) ? 1 : 0 ) << std::endl;
	params.output << params.height << " " << params.width << std::endl;
	params.output << params.framerate << std::endl ;

	int k = 0;
	int i, j;

	vector<vector<int> >* frame = new vector<vector<int>>(params.height, vector<int>(params.width));

	while (k < params.frames){
		if (k % (params.framesIntermedios+1) == 0 ){
			//Cargo la imagen actual
			for (i=0; i < params.height; i++){
				for (j=0; j < params.width; j++){
				params.input >> (*frame)[i][j];
				}
			}
			imprimirFrame(params.output, *frame, params.height, params.width);
		}else{
			// tiro el frame
			for (i=0; i < params.height; i++){
				//ignoro la linea
				//params.input.ignore(std::numeric_limits<std::streamsize>::max(), '\n');
				for (j=0; j < params.width; j++){
					params.input >> (*frame)[i][j];
					}
				}
			}
		k++;
	}
	delete frame;
}
\end{lstlisting}

\subsubsection{Algoritmo para calcular ECM} \label{ecmalgo}

\lstset{language=C++, breaklines=true, basicstyle=\footnotesize}
\begin{lstlisting}[frame=single]
double ecm (Parametros &params, Parametros &params2){
	assert(params.height == params2.height);
	assert(params.width == params2.width );
	int frames = (params.frames <= params2.frames) ? params.frames : params2.frames;
	double promedio = 0;
	double ecm;
	double max = 0;
	int i,j,k;
	
	vector<vector<double> >* frameA = new vector<vector<double>>(params.height, vector<double>(params.width));
	vector<vector<double> >* frameB = new vector<vector<double>>(params.height, vector<double>(params.width));

	for (k=0; k< 5; k++){
		//cargo Frame A
		for (i=0; i < params.height; i++){
			for (j=0; j < params.width; j++){
			params.input >> (*frameA)[i][j];
			}
		}
		//cargo Frame B
		for (i=0; i < params.height; i++){
			for (j=0; j < params.width; j++){
			params2.input >> (*frameB)[i][j];
			}
		}
		ecm = 0;
		for (i=0; i < params.height; i++){
			for (j=0; j < params.width; j++){
				ecm +=  pow ( abs( (*frameA)[i][j] - (*frameB)[i][j] ), 2);
			}
		}
		promedio += ecm; //sumo el ecm sin haberlo dividido por el mn , lo divido al final una vez sumados todos
		ecm = ecm / (params.width * params.height);
		//Imprimo ECM del frame k
		cout << "ECM frame " << k << ": " << ecm << std::endl ;
  		max = (ecm > max) ? ecm : max ;
		
	}
	promedio = promedio / (params.width * params.height);
	promedio = promedio / frames;
	params.output << "ECM promedio: " << promedio << std::endl;
	params.output << "Maximo error obtenido: " << max << std::endl;
	return promedio;

}
\end{lstlisting}

\newpage

\subsection{Apéndice B: Resultados cuantitativos sobre ECM y PSNR} \label{cuant}

La siguiente tabla muestra el ECM promedio obtenido con cada algoritmo para cada input, según la cantidad de frames internos (FI) eliminados: 

\bigskip
\captionof{table}{ECM: Promedios}
\begin{tabular}{| c | c | c | c | c | c | c | c |} 
\hline
\textbf{Método} & \textbf{FI} & \textbf{cupcake} & \textbf{perro} & \textbf{morocho} & \textbf{tenis} & \textbf{bebes} & \textbf{fideos} \\ 
\hline
Vecino Más Cercano & 2 & 45.0112 & 314.75 & 83.308 & 30.9356 & 120.327 & 61.5792 \\ 
\hline
Vecino Más Cercano & 4 & 75.0243 & 571.526 & 122.311 & 60.5956 & 201.26 & 116.639 \\ 
\hline
Vecino Más Cercano & 6 & 91.8027 & 762.867 & 203.109 & 90.501 & 264.514 & 165.972 \\ 
\hline
Interpolación Lineal & 2 & 29.8591 & 229.528 & 51.6138 & 19.3621 & 86.5087 & 43.6452 \\ 
\hline
Interpolación Lineal & 4 & 53.6601 & 427.874 & 92.725 & 40.4578 & 153.048 & 86.1893 \\ 
\hline
Interpolación Lineal & 6 & 75.7466 & 585.446 & 146.864 & 63.0961 & 209.167 & 125.779 \\ 
\hline
Splines (tamBloque=4) & 2 & 35.3717 & 256.704 & 65.2409 & 20.8447 & 98.4989 & 48.9282 \\ 
\hline
Splines (tamBloque=4) & 4 & 62.177 & 488.344 & 98.4069 & 43.6463 & 179.913 & 98.9888 \\ 
\hline
Splines (tamBloque=4) & 6 & 85.1963 & 664.815 & 162.418 & 68.5048 & 242.925 & 142.752 \\ 
\hline
Splines (tamBloque=8) & 2 & 33.5335 & 286.05 & 76.3302 & 22.4604 & 108.7 & 54.142 \\ 
\hline
Splines (tamBloque=8) & 4 & 66.3889 & 530.388 & 111.826 & 47.4056 & 189.55 & 106.137 \\ 
\hline
Splines (tamBloque=8) & 6 & 95.1269 & 716.159 & 178.607 & 73.7002 & 258.044 & 152.731 \\ 
\hline
Splines (tamBloque=16) & 2 & 37.763 & 299.482 & 80.32 & 23.8345 & 125.534 & 58.6143 \\ 
\hline
Splines (tamBloque=16) & 4 & 73.8393 & 564.792 & 114.907 & 50.9056 & 218.498 & 116.035 \\
\hline
Splines (tamBloque=16) & 6 & 98.9849 & 750.186 & 185.017 & 77.5372 & 265.835 & 159.943 \\
\hline
\end{tabular}

\bigskip

La siguiente tabla muestra el ECM máximo obtenido con cada algoritmo para cada input, según la cantidad de frames internos (FI) eliminados:

\bigskip
\captionof{table}{ECM: Máximos}
\begin{tabular}{| c | c | c | c | c | c | c | c |} 
\hline
\textbf{Método} & \textbf{FI} & \textbf{cupcake} & \textbf{perro} & \textbf{morocho} & \textbf{tenis} & \textbf{bebes} & \textbf{fideos} \\ 
\hline
Vecino Más Cercano & 2 & 3299.59 & 1649.76 & 5725.87 & 166.118 & 2744.17 & 254.567  \\
\hline
Vecino Más Cercano & 4 & 3302.67 & 1992.46 & 2098.77 & 302.99 & 2737.22 & 390.933 \\
\hline
Vecino Más Cercano & 6 & 316.587 & 2340.72 & 5714.16 & 419.074 & 2064.46 & 447.62 \\
\hline
Interpolación Lineal & 2 & 1474.74 & 1163.21 & 2551.35 & 112.955 & 1626.55 & 154.708 \\
\hline
Interpolación Lineal & 4 & 1316.88 & 1359.27 & 1492.78 & 202.953  & 1542.88 & 253.57 \\
\hline
Interpolación Lineal & 6 & 785.545 & 1610.4 & 3166.05 & 272.691 & 1552.01 & 316.921 \\
\hline
Splines (tamBloque=4) & 2 & 1959.89 & 1298.27 & 3358.05 & 130.554 & 1623.07 & 61.805 \\
\hline
Splines (tamBloque=4) & 4 & 1703.56 & 1490.22 & 1435.91 & 230.352 & 1784.34 & 325.296 \\
\hline
Splines (tamBloque=4) & 6 & 1037.42 & 2021.49 & 2845.6 & 316.764 & 2132.5 & 369.742 \\
\hline
Splines (tamBloque=8) & 2 & 1060.74 & 1340.14 & 3527.96 & 184.545 & 1970.12 & 244.584 \\
\hline
Splines (tamBloque=8) & 4 & 1703.56 & 1530.92 & 1361.28 & 305.854 & 2445.58 & 375.793 \\
\hline
Splines (tamBloque=8) & 6 & 1123.97 & 2808.73 &3158.33 & 422.622 & 1734.64 & 446.351 \\
\hline
Splines (tamBloque=16) & 2 & 1305.78 & 1913.96 & 3358.05 & 353.354 & 2009.64 & 445.129 \\
\hline
Splines (tamBloque=16) & 4 & 1793.46 & 2764.92 & 1397.55 & 674.102 & 1806.3 & 612.757 \\
\hline
Splines (tamBloque=16) & 6 & 1037.42 & 2985.46 & 2950.02 & 701.227 & 1766.2 & 572.348 \\
\hline
\end{tabular}

\bigskip

\newpage

La siguiente tabla muestra el PSNR promedio obtenido con cada algoritmo para cada input, según la cantidad de frames internos (FI) eliminados:

\bigskip
\captionof{table}{PSNR: Promedios}
\begin{tabular}{| c | c | c | c | c | c | c | c |} 
\hline
\textbf{Método} & \textbf{FI} & \textbf{cupcake} & \textbf{perro} & \textbf{morocho} & \textbf{tenis} & \textbf{bebes} & \textbf{fideos} \\ 
\hline
Vecino Más Cercano & 2 & 32.5224 &  22.2815 & 39.1805 &  35.4634 & 34.6831 & 29.0771 \\
\hline
Vecino Más Cercano & 4 & 30.5887 &  20.5858 & 33.6034 &  33.3133 & 32.5697 & 27.2282 \\
\hline
Vecino Más Cercano & 6 & 29.4282 &  19.5938 & 31.3594 &  31.9285 & 31.0408 &  26.0286 \\
\hline
Interpolación Lineal & 2 & 34.1298 &  23.6189 & 34.4392 &  38.4812 & 36.8022 &   30.2675 \\
\hline
Interpolación Lineal & 4 & 31.7302 &  21.6657 & 32.8016 &  35.2684 & 34.0206 &  28.3235 \\
\hline
Interpolación Lineal & 6 & 30.3006 & 20.5687  & 31.1006 &  33.7438 & 32.1325 & 26.998 \\
\hline
Splines (tamBloque=4) & 2 & 33.8389 &  23.1742 & 33.4144 &  38.7222 & 36.2412 & 30.1813 \\
\hline
Splines (tamBloque=4) & 4 & 31.4883 &  21.0954 & 32.7601 &  35.3889 & 33.3246 &  27.7709 \\
\hline
Splines (tamBloque=4) & 6 & 29.7734 &  20.0804 & 30.7256 & 33.4113  & 31.4074 & 26.5067 \\
\hline
Splines (tamBloque=8) & 2 & 33.3273 & 22.7614  & 32.4942 &  38.3841 & 35.577 &  29.8654 \\
\hline
Splines (tamBloque=8) & 4 & 31.1883 &  20.774 & 31.84 &  34.8538 & 32.639 &  27.4856 \\
\hline
Splines (tamBloque=8) & 6 & 29.4519 &  19.7952 & 30.08 &  32.8521 & 30.3118 & 26.1694 \\
\hline
Splines (tamBloque=16) & 2 & 33.114 &  22.6479 & 32.3172 &  38.0825 & 33.9903 & 29.6349  \\
\hline
Splines (tamBloque=16) & 4 & 30.6803 &  20.5192 & 31.6833 &  34.4156 & 29.5255  & 27.004 \\
\hline
Splines (tamBloque=16) & 6 & 29.1705 &  19.4903 & 29.9807 &  32.4446 & 29.1436 & 25.9149 \\
\hline
\end{tabular}

\bigskip