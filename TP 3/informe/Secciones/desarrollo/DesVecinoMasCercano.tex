\par Para comenzar con el desarrollo del algoritmo de Vecinos más Cercanos, nos planteamos la idea principal del algoritmo y qu\'e decisi\'on tomaremos en caso de que un frame con igual distancia a ambos vecinos. En esa circunstancia le daremos el valor de su siguiente vecino.

\par El algoritmo se basa en un ciclo que itera sobre la cantidad de frames del video original. En cada iteraci\'on lo que hacemos es tomar el frame previo e imprimirlo en el archivo de salida $F / 2$ veces, siendo $F$ = $"$cantidad de frames intermedios que se quieren agregar$"$. Con esto logramos que los vecinos que se encuentran de la mitad hacia el lado del vecino inferior tomen su valor. Mientras que luego imprimimos otros $F / 2$ frames pero con los valores del vecino superior. En caso de que $F / 2$ no sea un valor entero, tomaremos ese valor central como más cercano al superior.