\par El algoritmo de splines genera un spline por cada posición de un pixel de un frame, de manera que para generar un nuevo frame intermedio se utilizan todos los splines generados. A su vez, se divide la cantidad de frames del video en bloques. Esto se hace ya que no tiene sentido calcular los splines tomando todos los frames del video ya que pueden ser muy diferentes o pertenecer a distintas tomas, por ejemplo. Además cabe destacar que al aumentar el tamaño de los bloques aumenta la complejidad espacial del algoritmo, ya que necesita el valor de los píxeles de todos los frames del bloque para poder calcular las incógnitas.  La cantidad mínima de frames por bloques es 4 ya que se necesita de 4 puntos como mínimo para poder calcular el polinomio cúbico.
\par El algoritmo entonces, itera sobre cada uno de los bloques en los que se divide el video (según el tamaño de bloque ingresado por parámetro, que debe ser mayor o igual a 4) para calcular los splines correspondientes. En cada iteración, se calcula la matriz para calcular los $c_i$ y el vector $res$ para cada posición $[i,j]$ de los frames, como fue explicado en la sección de introducción teórica. Como fue mencionado anteriormente, esta matriz es estrictamente diagonal dominante y por lo tanto inversible, por lo que se puede calcular la solución al sistema sin problemas. Para calcularla, el algoritmo primero diagonaliza la matriz aprovechando el hecho de que sea una matriz tridiagonal y luego simplemente calcula los $c_i$ despejando los valores.
\par Una vez obtenidos los $c_i$ para cada spline, se calculan los $d_i$ y los $b_i$ y se calculan los valores de los píxeles de los nuevos frames intermedios usando los splines generados. Para eso, se evalúan los polinomios generados en los $x$ correspondientes a la distancia entre el nuevo frame y el $x_i$ anterior. 
