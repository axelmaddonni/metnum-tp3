% Explicación básica sobre interpolación y splines ??

\par Para interpolar los frames de nuestro video, consideraremos una tabla de puntos [$x_i$, $y_i$] con $i=0,1,...,n$ por cada pixel de un frame del tamaño ingresado por parámetro, con $n=\#frames\ originales$. En nuestro modelo cada $x_{i}$ representará el número de frame correspondiente, mientras que el $y_{i}$ representará el valor del píxel en dicho frame. Entonces computaremos un spline por cada píxel que luego usaremos para formar los frames intermedios que buscamos agregar al video.

\par Cada $x_i$ tomará en cuenta los frames intermedios, por ejemplo, si tomamos una entrada con framesIntermedios = 2 y cantidad de frames originales = 2, $x_0=0$(frame inicial), $x_1=3$ (frame final), y luego con el spline obtenido con el algoritmo calcularemos los valores intermedios tomando $x=1$ y $x=2$ respectivamente. Otra opción sería tomar $x_0=0$ y $x_1=1$, y calcular los frames Intermedios tomando $x=1/3$ y $x=2/3$, pero para trabajar con valores enteros y evitar perder precisión, como hemos observado en el experimento [INGRESAR NUMERO DE EXPERIMENTO SI LLEGAMOS A HACERLO], usaremos la primera alternativa. 
\begin{equation*}\label{calculo de xi}
x_i = framesIntermedios * i + i
\end{equation*}
\par A continuación se explica el método de interpolación por Splines o Trazados Cúbicos para una tabla de puntos [$x_i$, $y_i$], es decir, en nuestro modelo, un pixel por cada frame ubicados en la misma posición. El algoritmo presentado computa los splines para los demás pixels de manera análoga. En la sección Desarrollo se especifican cuestiones sobre el funcionamiento del algoritmo y la paralelización deĺ cálculo de los splines.
\par Como se menciona anteriormente, comenzamos con una tabla con $n+1$ puntos y $n$ intervalos entre ellos. Un spline de interpolación cúbica es una curva continua definida por partes, que pasa por todos los valores de la tabla. Hay un polinomio cúbico por cada intervalo, cada uno con sus coeficientes de la siguiente manera:
\begin{equation*}\label{splines}
S_i(x) = a_i + b_i(x-x_i) + c_i(x-x_i)^2 + d_i(x-x_i)^3 \ \ \ \forall x\ \in\ [x_i, x_{i+1}]
\end{equation*}
\par Estos polinomios juntos se denominan spline $S(x)$.
\par Como hay $n$ intervalos y 4 coeficientes para cada polinomio, requerimos $4n$ parametros para poder definir el spline, es decir, es necesario encontrar $4n$ condiciones independientes que cumpla el spline. Obtenemos 2 condiciones por cada intervalo para que el polinomio fitee los valores de la tabla en ambos extremos del intervalo y resulte interpolante:
\begin{equation}\label{cond1}
S_i(x_i) = y_i  \ \ \ \forall i=0, ..., n-1 \tag{cond1}
\end{equation}
\begin{equation}\label{cond2}
S_i(x_{i+1}) = y_{i+1} \ \ \ \forall i=0,... n-1 \tag{cond2}
\end{equation}
\par Notar que estas condiciones resultan en que la función definida por el spline sea continua. Todavía necesitamos $2n$ condiciones. Pedimos que las derivadas primeras y segundas de la función también sean continuas:
\begin{equation}\label{cond3}
S_{i+1}'(x_{i+1}) = S_{i}'(x_{i+1}) \ \ \  \forall  i=0, ..., n-2 \tag{cond3}
\end{equation}
\begin{equation}\label{cond4}
S_{i+1}''(x_{i+1}) = S_{i}''(x_{i+1}) \ \ \  \forall i=0, ..., n-2 \tag{cond4}
\end{equation}
\par Estas últimas dos condiciones aplican para $i=0, ..., n-2$, resultando en 2(n-1) ecuaciones. Por lo que necesitamos dos condiciones más para completar el spline. Existen dos opciones:

\begin{itemize}
\item Spline natural : $S''_0(x_0) =  S''_{n-1}(x_n) = 0$
\item Spline sujeto : $S'_0(x_0) = f'(x_0)\ y\ S'_{n-1}(x_n) = f'(x_n) $
\end{itemize}

\par Con estas $4n$ ecuaciones, es posible reducir las condiciones a un sistema triangular con los coeficientes $c_i$ como las incógnitas. Procedemos a reducir las ecuaciones:

% Reducir las 4 ecuaciones

% Armar el sistema triangular

% Calculo de los demás coeficientes

\setlength{\belowdisplayskip}{0pt} \setlength{\belowdisplayshortskip}{0pt}
\setlength{\abovedisplayskip}{0pt} \setlength{\abovedisplayshortskip}{0pt}

\begin{align*}
\intertext{De la condición \eqref{cond1}: } 
S(x_i) = & f(x_i)				& \forall i=0, ..., n-1 \\
a_i + b_i(x_i-x_i) + c_i(x_i-x_i)^2 + d_i(x_i-x_i)^3  = & f(x_i) 	& \forall i=0, ..., n-1 \\
\Aboxed{a_i = & f(x_i)} 	& \forall i=0, ..., n-1 \\
\end{align*}


\begin{align*}
\intertext{Expresemos \eqref{cond2} restando en 1 los subíndices: }
y_i = & S_{i-1}(x_i)				& \forall i=1, ..., n \\
y_i = & a_{i-1} + b_{i-1}h_{i-1} + c_{i-1}h_{i-1}^2 +  d_{i-1}h_{i-1}^3 & \forall i=1, ..., n \\
\text{Como } a_{i-1} = y_{i-1} \ \forall i=1, ..., n \text{ por (1): } \\
\Aboxed{y_i = & y_{i-1} + b_{i-1}h_{i-1} + c_{i-1}h_{i-1}^2 +  d_{i-1}h_{i-1}^3} & \forall i=1, ..., n \\
\text{donde } h_{i-1} = (x_i-x_{i-1}) \\
\end{align*}

\begin{align*}
\intertext{Calculamos la derivada de un spline: }
S'_i(x) = & b_i + 2c_i(x-x_i) + 3d_i(x-x_i)^2  & \forall i=0, ..., n-1 \\
\text{Tomando } x = x_i:  \\
S'_i(x_i) = & b_i	 & \forall i=0, ..., n-1 \\
\intertext{Expresamos \eqref{cond3} restando en 1 los subíndices : }
S'_{i-1}(x_i) = & S'_i(x_i) 	 & \forall i= 1, ..., n-1 \\
\intertext{ Combinándolas tenemos: } 
b_i =  S'_i(x_i) = S'_{i-1}(x_i) = & b_{i-1} + 2c_{i-1}h_{i-1} + 3d_{i-1}h_{i-1}^2 & \forall i=1, ..., n-1 \\
\text{Tomando } b_n =  S'_{n-1}(x_n) = & b_{n-1} + 2c_{n-1}h_{n-1} + 3d_{n-1}h_{n-1}^2 & \\
\Aboxed{b_i = & b_{i-1} + 2c_{i-1}h_{i-1} + 3d_{i-1}h_{i-1}^2} & \forall i=1, ..., n \\
\text{donde } h_{i-1} = (x_i-x_{i-1}) \\
\end{align*}

\begin{align*}
\intertext{Ahora calculamos la segunda derivada: }
S''_i(x) = & 2c_i + 6d_i(x-x_i)   & \forall i=0, ..., n-1 \\
S''_i(x_i) = & 2c_i & \forall i=0, ..., n-1 \\
\intertext{Expresamos \eqref{cond4} restando en 1 los subíndices: } 
S''_{i-1}(x_i) = & S''_i(x_i) 	 & \forall i= 1, ..., n-1 \\
\intertext{ Combinándolas tenemos: } 
2c_i =  S''_i(x_i) = S''_{i-1}(x_i) = & 2c_{i-1} + 6d_{i-1}h_{i-1}	 & \forall i= 1, ..., n-1 \\
\text{Tomando } c_n = & S''_{n-1}(x_n) = 2c_{n-1} + 6d_{n-1}h_{n-1} \\
\Aboxed{2c_i = & 2c_{i-1} + 6d_{i-1}h_{i-1}} & \forall i= 1, ..., n \\
\text{donde } h_{i-1} = (x_i-x_{i-1}) \\
\end{align*}

% Armar el sistema triangular

\par Ahora podemos armar el sistema para calcular los $c_i$. Los siguientes cálculos permiten expresar los $c$ en términos de $h$ e $y$, eliminándo las $b$ y $d$ de las ecuaciones:

\par Tenemos:

\begin{align}
\setcounter{equation}{0}
y_i =& y_{i-1} + b_{i-1}h_{i-1} + c_{i-1}h_{i-1}^2 +  d_{i-1}h_{i-1}^3  & \forall i= 1, ..., n \tag{eq1} \label{eq1}\\
b_i =& b_{i-1} + 2c_{i-1}h_{i-1} + 3d_{i-1}h_{i-1}^2 & \forall i= 1, ..., n  \tag{eq2} \label{eq2}\\
2c_i =& 2c_{i-1} + 6d_{i-1}h_{i-1}  & \forall i= 1, ..., n \tag{eq3} \label{eq3}
\end{align}

%\numberthis \label{eqn}

\begin{align*}
\intertext{ Despejamos $b_{i-1}$ de \eqref{eq1}: } 
b_{i-1} = &  \frac{y_i-y_{i-1}}{h_{i-1}} - c_{i-1}h_{i-1} - d_{i-1}h_{i-1}^2  & & \forall i= 1, ..., n & \tag{eq4} \label{eq4} \\
b_{i} = &  \frac{y_{i+1}-y_{i}}{h_{i}} - c_{i}h_{i} - d_{i}h_{i}^2  & & \forall i= 0, ..., n-1 & \tag{eq5} \label{eq5} \\
\end{align*}

\begin{align*}
\intertext{ Escribimos \eqref{eq3} como: } 
2c_i-2c_{i-1} = & 6d_{i-1}h_{i-1}		  & & \forall i= 1, ..., n \\
\text{multiplicamos por } h_{i-1} \text{ y dividimos por 2 ambos lados: } \\
h_{i-1}(c_i-c_{i-1}) = & 3d_{i-1}h_{i-1}^2 		 & & \forall i= 1, ..., n  & \tag{eq6} \label{eq6} \\
\end{align*}

\begin{align*}
\intertext{De la \eqref{eq2} podemos despejar } 
3d_{i-1}h_{i-1}^2 = & b_i - b_{i-1} - 2c_{i-1}h_{i-1} & & \forall i= 1, ..., n  \tag{eq7} \label{eq7} \\
\intertext{Reemplazamos \eqref{eq7} en el lado derecho de \eqref{eq6}: } 
h_{i-1}(c_i-c_{i-1}) = & b_i - b_{i-1} - 2c_{i-1}h_{i-1} & & \forall i= 1, ..., n \\
2c_{i-1}h_{i-1} + h_{i-1}(c_i-c_{i-1}) = & (b_i-b_{i-1}) & & \forall i= 1, ..., n \\
h_{i-1}(c_i+c_{i-1}) = & (b_i-b_{i-1}) & & \forall i= 1, ..., n \tag{eq8} \label{eq8}
\end{align*}

\begin{align*}
\intertext{Reemplazamos \eqref{eq4} y \eqref{eq5} en \eqref{eq8}: } 
h_{i-1}(c_i+c_{i-1}) =  & (\frac{y_{i+1}-y_{i}}{h_{i}} - c_{i}h_{i} - d_{i}h_{i}^2) - (\frac{y_i-y_{i-1}}{h_{i-1}} - c_{i-1}h_{i-1} - d_{i-1}h_{i-1}^2)  \\
&   \forall i= 1, ..., n-1 \\
c_{i}h_{i} - c_{i-1}h_{i-1} + h_{i-1}(c_i+c_{i-1}) =  & (\frac{y_{i+1}-y_{i}}{h_{i}} - d_{i}h_{i}^2) - 
(\frac{y_i-y_{i-1}}{h_{i-1}} - d_{i-1}h_{i-1}^2) \\
&   \forall i= 1, ..., n-1  \\
c_{i}h_{i} +  c_ih_{i-1} =  & (\frac{y_{i+1}-y_{i}}{h_{i}} -  \frac{y_i-y_{i-1}}{h_{i-1}}) + (d_{i-1}h_{i-1}^2  - d_{i}h_{i}^2) \\
&   \forall i= 1, ..., n-1 \tag{eq9} \label{eq9} 
\end{align*}

\begin{align*}
\intertext{De \eqref{eq6} podemos despejar: }
d_{i-1}h_{i-1}^2 = & \frac{1}{3}h_{i-1}(c_i-c_{i-1}) & \forall i= 1, ..., n \tag{eq10} \label{eq10}\\
d_{i}h_{i}^2 = & \frac{1}{3}h_{i}(c_{i+1}-c_{i}) & \forall i= 0, ..., n-1 \tag{eq11} \label{eq11}\\
\end{align*}

\begin{align*}
\intertext{Reemplazo \eqref{eq10} y \eqref{eq11} en \eqref{eq9}:}
c_{i}h_{i} +  c_ih_{i-1} = (\frac{y_{i+1}-y_{i}}{h_{i}}  -\frac{y_i-y_{i-1}}{h_{i-1}}) + \frac{1}{3}\Big((h_{i-1}(c_i-c_{i-1})  - h_{i}(c_{i+1}-c_{i})\Big) & \forall i= 1, ..., n-1 \\
\intertext{multiplicamos por 3 en ambos lados y movemos los términos con c a la izquierda}
3c_{i}h_{i} +  3c_ih_{i-1} - \Big(h_{i-1}(c_i-c_{i-1})  - h_{i}(c_{i+1}-c_{i})\Big) =  3\Big(\frac{y_{i+1}-y_{i}}{h_{i}}  -\frac{y_i-y_{i-1}}{h_{i-1}}\Big) & \forall i= 1, ..., n-1 \\
c_{i-1}h_{i-1} + 2c_ih_{i-1} + 2c_ih_i + c_{i+1}h_i =  3\Big(\frac{y_{i+1}-y_{i}}{h_{i}} - \frac{y_i-y_{i-1}}{h_{i-1}}\Big)  & \forall i= 1, ..., n-1 \\
\Aboxed{c_{i-1}h_{i-1} + 2c_i(h_{i-1} + h_i) + c_{i+1}h_i =  3\Big(\frac{y_{i+1}-y_{i}}{h_{i}}  -\frac{y_i-y_{i-1}}{h_{i-1}}\Big)} & \forall i= 1, ..., n-1 \tag{eq12} \label{eq12}
\end{align*}

\par Podemos escribir el sistema de ecuaciones resultante para calcular los $c_i$ usando la siguiente matriz tridiagonal: 

% ARREGLAR MATRIZ, AGREGAR PRIMERA Y ULTIMA FILA COMENTANDO LA JUSTIFICACIÓN POR SPLINES NATURL O LIBRE
% AGREGAR NATURAL VS SUJETO
% REVISAR TIPEO DE ECUACIONES

$\begin{array}{l} \begin{bmatrix} 
1 & 0 & 0 & 0 & 0 & 0 \\
h0 & 2(h_0+h_1) & h_1 & \text{} & \text{} & \text{} \\ 
0 & h_1 & 2(h_1+h_2) & h_2 & \text{} & \text{} \\ 
\text{} & \ddots & \ddots & \ddots & \ddots & \text{} \\ 
\text{} & \text{} & h_{n-3} & 2(h_{n-3}+h_{n-2}) & h_{n-2} & 0 \\
\text{} & \text{} & \text{} & h_{n-2} & 2(h_{n-2}+h_{n-1}) & h_{n-1} \\
0 & 0 & 0 & 0 & 0 & 1 \\
\end{bmatrix} 
\begin{bmatrix} c_0 \\ c_1 \\ c_2 \\ \vdots \\ c_{n-2} \\ c_{n-1} \\ c_n \\ \end{bmatrix} 
= 3 \begin{bmatrix} 0 \\ \frac{y_{2}-y_{1}}{h_{1}}  -\frac{y_1-y_{0}}{h_{0}} \\ \vdots \\ \vdots \\ \vdots \\ \frac{y_{n}-y_{n-1}}{h_{n-1}}  -\frac{y_{n-1}-y_{n-2}}{h_{n-2}} \\ 0 \\ \end{bmatrix} \end{array}$

\par donde la primera y última de la matriz corresponden a las condiciones generadas por el hecho de que el spline sea Libre. Un spline libre cumplía: 

\begin{align*}
S''(x_0) = S''_0(x_0) = & 0 \\
2c_0 + 6d_0(x_0 - x_0) = & 0 \\
c_0 = & 0 & \text{Primera fila}
\end{align*}

\begin{align*}
S''(x_n) = S''_{n-1}(x_n) = & 0 \\
\text{pero } S''_{n-1}(x_n) = & c_n \\
\Rightarrow c_n = & 0 & \text{Última fila}
\end{align*}


\par Una vez obtenidos los $c_i$, podemos calcular los $b_i$ usando la \eqref{eq5} y los $d_i$ usando \eqref{eq3}. La forma en que resolvemos el sistema está detallada en la sección Desarrollo. Notar que la matriz resultante es estrictamente diagonal dominante, por lo que es inversible y tiene solución.



